% Compile with:
% latexmk -lualatex -pvc -interaction=nonstopmode 20220629_MICResearchDay.tex
%\documentclass[aspectratio=169,draft]{beamer}
\documentclass[aspectratio=169]{beamer}
\usetheme{UniBern}

\title{micro-CT imaging across scales and faculties}
\author{David Haberthür}
\date{June 29, 2022 | \href{https://www.mic.unibe.ch/events/mic_research_day_2022}{MIC Research Day 2022}}

%\includeonlyframes{current}
%then....
%\begin{frame}[label=current]
%\end{frame}

\usepackage[detect-all=true,
	per-mode=symbol,
	per-symbol=/]{siunitx}
\usepackage{xspace}
\usepackage{gitinfo2}
\usepackage[backend=biber,
	style=numeric,
	url=false,
	maxnames=1,
	sorting=none,
	]{biblatex}
\addbibresource{~/P/Documents/library.bib}
\usepackage{ccicons}
\usepackage{animate}
\usepackage{tikz}
	\usetikzlibrary{spy}
	\tikzset{shadowed/.style={preaction={transform canvas={shift={(1pt,-1pt)}},draw=ubRed}}}
\usepackage{shadowtext} % for the shadowed scalebar
	\shadowoffset{1pt}
	\shadowcolor{ubRed}
\usepackage[absolute,overlay]{textpos} % for the \source* command
\usepackage{fontawesome5}
\usepackage{emoji}
\usepackage{listings}
	\lstset{frame=single,
	basicstyle=\tiny\ttfamily
	}
\usepackage{booktabs}
\usepackage{multirow}
\usepackage{colortbl}
\usepackage{adjustbox}
\usepackage{microtype}

% Some often used abbreviations/commands
\newcommand{\everyframe}{25} % use only every nth frame for the animations
\newcommand{\imwidth}{\linewidth}% set global image width
\newcommand{\imheight}{0.8\paperheight}% set global image height
\newlength\imagewidth% needed for scalebars
\newlength\imagescale% needed for scalebars
\newcommand{\uct}{\textmu CT\xspace}% make our life easier
\newcommand{\eg}{e.\,g.\xspace}%
\newcommand{\ie}{i.\,e.\xspace}%

% Define complementary colors to ubRed
\definecolor{ubRedComplementary1}{HTML}{00a1e6}
\definecolor{ubRedComplementary2}{HTML}{00e645}

% Acknowledge images just below them
% Based on https://tex.stackexchange.com/a/282637/828
\newcommand{\source}[2]{%
	% Print out (short) link under image, with small text
	\raisebox{-1.618ex}{%
		\makebox[0pt][r]{%
			\scriptsize\href{http://#1}{#1} #2%
			}%
		}%
	}%
\newcommand{\sourcecite}[2]{%
	% Cite (an image from) a reference
	\raisebox{-1.618ex}{%
		\makebox[0pt][r]{%
			\scriptsize From \cite{#1}, #2%
			}%
		}%
	}%
\newcommand{\sourcelink}[3]{%
	% Make the source command an \href{link}{text}
	\raisebox{-1.618ex}{%
		\makebox[0pt][r]{%
			\scriptsize\href{http://#1}{#2}, #3%
			}%
		}%
	}%

% Define us a custom footer *with* progress bar, based on https://tex.stackexchange.com/a/59749/828
\makeatletter
\def\progressbar@progressbar{} % the progress bar
\newcount\progressbar@tmpcounta% auxiliary counter
\newcount\progressbar@tmpcountb% auxiliary counter
\newdimen\progressbar@pbht %progressbar height
\newdimen\progressbar@pbwd %progressbar width
\newdimen\progressbar@rcircle % radius for the circle
\newdimen\progressbar@tmpdim % auxiliary dimension
\progressbar@pbwd=0.85\linewidth
\progressbar@rcircle=1.5pt
\def\progressbar@progressbar{%
	\progressbar@tmpcounta=\insertframenumber
	\progressbar@tmpcountb=\inserttotalframenumber
	\progressbar@tmpdim=\progressbar@pbwd
	\multiply\progressbar@tmpdim by \progressbar@tmpcounta
	\divide\progressbar@tmpdim by \progressbar@tmpcountb
	\par%
	\begin{tikzpicture}%
		\draw[ubGrey] (0,0) -- ++ (\progressbar@pbwd,0);
		\draw[draw=ubRed,fill=ubGrey] (\the\dimexpr\progressbar@tmpdim-\progressbar@rcircle\relax,.5\progressbar@pbht) circle (\progressbar@rcircle);
	\end{tikzpicture}%
	\hfill bit.ly/MICRday\xspace|\xspace%
	v. \href{https://github.com/habi/Talk.2022.MICResearchDay/commit/\gitHash}{\gitAbbrevHash}\xspace|\xspace%
	p.\xspace\insertframenumber/\inserttotalframenumber%
	\hspace*{4ex}%
	\vspace{0.5ex}%
	%\par%
}
\addtobeamertemplate{footline}{}%
{%
	\begin{beamercolorbox}[wd=\paperwidth,center]{green}%
		\progressbar@progressbar%
	\end{beamercolorbox}%
}%
\makeatother

% Format bibliography for beamer
% http://tex.stackexchange.com/a/10686/828
\renewbibmacro{in:}{}
% http://tex.stackexchange.com/a/13076/828
\AtEveryBibitem{%
	\clearfield{journaltitle}
	\clearfield{pages}
	\clearfield{volume}
	\clearfield{number}
	\clearname{editor}
	\clearfield{issn}
	\clearfield{year}
}
% No parentheses around the (now empty) year: https://tex.stackexchange.com/a/147537/828
\renewcommand{\bibopenparen}{\addcomma\addspace}
\renewcommand{\bibcloseparen}{\addcomma\addspace}

% open in fullscreen
\hypersetup{pdfpagemode=FullScreen}

% Move the text down a bit
% THIS IS A BIG HACK, IT SHOULD BE FIXED IN THE TEMPLATE
\addtobeamertemplate{frametitle}{}{\vspace*{0.75em}}

\begin{document}
% No footline on the title page
% http://tex.stackexchange.com/a/18829/828 helps us to achieve that
{%
	\setbeamertemplate{footline}{}%
	\begin{frame}%
		\maketitle
	\end{frame}%
}

\begin{frame}
	\frametitle{\emph{Grüessech} from the \uct-group}
	\centering
	\includegraphics[width=\linewidth]{./images/team}
\end{frame}

\begin{frame}
	\frametitle{LOAFS}
	\uct
	
	Non-destructive imaging across scales
	
	Ich zeige nur \emph{uni-interne} Projekte, wir machen noch (viel) mehr	
\end{frame}

\begin{frame}
	\frametitle{Machinery at the Institute of Anatomy}
	\begin{columns}
		\begin{column}{0.333\linewidth}
			\includegraphics[width=\linewidth]{./images/1172}%
			\source{brukersupport.com}{}
		\end{column}
		\begin{column}{0.333\linewidth}
			\includegraphics<1>[width=\linewidth]{./images/1272}%
			\source{bruker.com/skyscan1272}{}
		\end{column}
		\begin{column}{0.333\linewidth}
			\includegraphics<1>[width=\linewidth]{./images/2214}%
			\source{bruker.com/skyscan2214}{}
		\end{column}							
	\end{columns}
\end{frame}

\begin{frame}
	\frametitle{Overview}
	% Get emojis from https://unicode.org/emoji/charts/emoji-list.html
	\centering
	\huge
	\emoji{horse} % or \emoji{unicorn} ?
	\emoji{peach}
	\emoji{fish} % or \emoji{tropical-fish} ?
	\emoji{mouse-face}  % or \emoji{mouse} ?
	\emoji{tooth}
	\emoji{nut-and-bolt}
	\emoji{comet}
\end{frame}

\begin{frame}
	\frametitle{Overview}
	\begin{columns}
		\begin{column}{0.309\linewidth}
			Scales
			\begin{itemize}
				\item Hoof
				\item Sakrum
				\item Fishes, large to small	
				\item Mouse brain (?)
				\item Teeth
				\item Implants (Straumann, e.g. no faculty)
				\item Chondrules
			\end{itemize}
		\end{column}
		\begin{column}{0.618\linewidth}
			Faculties
			\begin{itemize}
				\item Vetsuisse Faculty
				\item Faculty of Science
				\begin{itemize}
					\item Institute of Ecology and Evolution
					\item Space Research \& Planetary Sciences (WP)
				\end{itemize}	
				\item Faculty of Medicine
				\begin{itemize}
					\item Theodor Kocher Institute (TKI)
					\item School of Dental Medicine
					\item Institute of Anatomy
					\item Universitätsklinik für Orthopädische Chirurgie und Traumatologie
				\end{itemize}
			\end{itemize}
		\end{column}
	\end{columns}
\end{frame}

\begin{frame}
	\frametitle{Hoof}
	\centering
	\animategraphics[palindrome,height=\imheight,every=\everyframe]{24}{./movies/VETSUISSE/HorseLimb/Limb02/frames/video-0}{000}{300}
\end{frame}	

\begin{frame}
	\frametitle{Sakrum}
	\centering
	\animategraphics[palindrome,height=\imheight,every=\everyframe]{24}{./movies/Halm/Becken/Sakrum_C/frames/video-0}{000}{101}
\end{frame}	

\begin{frame}
	\frametitle{Thanks}
	Images from us
	
	Werbung für Workshop: COMULIS Training School: \url{https://www.ana.unibe.ch/continuing_education/comulis_training_school/}
\end{frame}

\begin{frame}
	\frametitle{References}
	% Make the references continuously smaller :)
	%\renewcommand*{\bibfont}{\small}
	%\renewcommand*{\bibfont}{\footnotesize}
	%\renewcommand*{\bibfont}{\scriptsize}
	%\renewcommand*{\bibfont}{\tiny}
%	\setbeamertemplate{bibliography item}{\insertbiblabel}
	\printbibliography
\end{frame}

\end{document}
